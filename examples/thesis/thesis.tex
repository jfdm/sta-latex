%% --------------------------------------------------------------- [ My Thesis ]
\documentclass[
%draft
final
]{sta-thesis}

\usepackage[style=alphabetic,natbib=true]{biblatex}
\usepackage{lipsum}

%% ---------------------------------------------------------------- [ Preamble ]

\ExecuteBibliographyOptions{maxnames=3,minnames=2}

%% ---------------------------------------------------------------- [ Metadata ]
\title{Ninja Thesis}
\author{Hamato Yoshi}
\email{hamato@yoshi.com}

\date{Compilation Date}

\address{%
First Line\\
Second Line\\
Third Line\\
\url{http://www.somewhere.com}
}

\university{University Name}
\logocmd{\includegraphics[scale=1.2]{./sta-logo.pdf}}

\degree{Doctor of Philosophy}

\keywordlist{keyword; keyword; keyword}

\bibliography{biblio.bib}
%% ------------------------------------------------------------------- [ Begin ]
\begin{document}
\maketitle 
%% ------------------------------------------------------------- [ Frontmatter ]
\frontmatter
%%
\thispagestyle{empty}
\begin{abstract}

\end{abstract}
\newpage
\begin{declaration}
I declare that the material submitted for assessment is my own work,
except where credit is explicitly given to others by citation or
acknowledgement.
\end{declaration}

\begin{copyrightnotice}
In submitting this project to the \emph{\theuniversity}, I give
permission for it to be made available for use in accordance with the
regulations of the \emph{\theuniversity}. I also give permission for
the title and abstract to be published and for copies of the report to
be made and supplied to any bona fide library or research worker, and
to be made available on the World Wide Web. 
\end{copyrightnotice}

%% -------------------------------------------------------------- [ Dedication ]
\begin{dedication}
Some dedication
\end{dedication}

%% --------------------------------------------------------------------- [ TOC ]
\tableofcontents

%% -------------------------------------------------------------- [ Mainmatter ]
\mainmatter

\chapter{Introduction}
\nocite{*}
\citeauthor{Shannon1946} wrote a cool paper \citep{Shannon1946}. The security of public key protocols has been studied before \citep{Dolev1983}. \citeauthor{Shamir1979} introduces in \citet{Shamir1979} how to share secrets.
There are many interesting papers outthere \cite{Needham1978,Rivest1978,Merkle1978,Diffie1976,Shannon1949,Ziv1977, Huffman1952, Hamming1950}

\begin{fncycomment}
This should only be seen in draft mode.
\end{fncycomment}

\begin{quote}
We few, we happy few, we band of brothers\anote[Bob]{Look Ma' a todo note};
For he to-day that sheds his blood with me
Shall be my brother; be he ne'er so vile,
This day shall gentle his condition:
And gentlemen in England now a-bed
Shall think themselves accursed they were not here,
And hold their manhoods cheap whiles any speaks
That fought with us upon Saint Crispin's day.
\end{quote}

\lipsum
\chapter{Method}

\lipsum

\begin{theorem}\label{th:a}
Let $a$ and $b$ be in $\mathbb{N}$. Then there exist integers $u$ and $v$, such that
\[
gcd(a,b) = ua+vb.
\]
In particular, if $a$ and $b$ are coprime, there exist integers $u$ and $v$, such tat
\[
ua+vb=1.
\]
\end{theorem}

\begin{lemma}\label{lem:a}
Let $d$ divide a product $ab$ and let the gcd of $d$ and $a$ be 1. Then $d$ divides $b$.
\end{lemma}

\begin{proof}
Since $\mathrm{gcd}(d,a)=1$, Theorem~\ref{th:a} implies that $xd + ya = 1$, for some integers $x$ and $y$.
So, $xdb + yab = b$. Since $d$ divides $ab$, it follows that $d$ also divides $xdb + yab$ which equals $b$.
\end{proof}

\begin{corollary}
let $p$ be prime and let $p$ divide $\prod_{i=1}^{k}a_{i}$, where $a_{i}\in\mathbb{Z}, 1\geq i\geq k$. Then $p$ divides at least one of the factors $a_{i}, 1\geq i\geq k$.
\end{corollary}

\begin{proof}
Use Lemma~\ref{lem:a} and induction on $k$.
\end{proof}

\noindent
A random labelled equation.
\begin{equation}\label{eq:a}
a_{i}\equiv a_{j} \pmod{m}\Rightarrow i=j
\end{equation}

\begin{theorem}[Fermat's Little Theorem]
Let $p$ be a prime number and let $a$ be any integer. Then
\[
a^{p}\equiv a\pmod{p}
\]
\end{theorem}

\begin{definition}
A function $f:\mathbb{N}\rightarrow\mathbb{N}$ is said to be \emph{multiplicative}, if for every pair of positive integers $m$ and $n$
\[
\mathrm{gcd}(m,n)=1\Rightarrow f(m,n)=f(m)f(n)
\]
\end{definition}

\begin{lemma}
Euler's Totient function $\phi(m)$ is multiplicative.
\end{lemma}

\begin{example}
An example
\end{example}

\begin{exercise}
An exercise
\end{exercise}

\begin{problem}
A problem
\end{problem}

\begin{solution}
A solution
\end{solution}

\begin{proposition}
A proposition
\end{proposition}


\begin{definition}
A definition
\end{definition}

\begin{note}
\lipsum[1]
\end{note}

\begin{remark}
\lipsum[1]
\end{remark}

\chapter{Results}

\lipsum
\chapter{Discussion}

\begin{quote}
We few, we happy few, we band of brothers\anote[Bob]{Look Ma' a todo note};
For he to-day that sheds his blood with me
Shall be my brother; be he ne'er so vile,
This day shall gentle his condition:
And gentlemen in England now a-bed
Shall think themselves accursed they were not here,
And hold their manhoods cheap whiles any speaks
That fought with us upon Saint Crispin's day.
\end{quote}

\lipsum[2-4]

\chapter{Conclusion}

\lipsum

%% ------------------------------------------------------------- [ Back Matter ]
\backmatter 
%% ------------------------------------------------------------ [ Bibliography ]
\printbibliography
%% ----------------------------------------------------------- [ List o' Stuff ]
\listoftheorems[ignore={note,remark}]

\end{document}
%% --------------------------------------------------------------------- [ EOF ]
