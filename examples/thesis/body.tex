\usepackage[style=alphabetic,natbib=true]{biblatex}
\usepackage{lipsum}

%% ---------------------------------------------------------------- [ Preamble ]

\ExecuteBibliographyOptions{maxnames=3,minnames=2}

% You can customise based on memoir here.

%% ---------------------------------------------------------------- [ Metadata ]
\title{Ninja Thesis}
\author{Hamato Yoshi}
\email{hamato@yoshi.com}

\date{\origdate\today}

\address{%
First Line\\
Second Line\\
Third Line\\
\url{http://www.somewhere.com}
}

\university{University Name}

% For copyright reasons I do not include the university logo.
\logocmd{\includegraphics[height=0.3\textwidth]{logo.png}}

\wordcount{80000}
\degree{Doctor of Philosophy}

\keywordlist{keyword; keyword; keyword}
\startdate{May}{2011}
\finishdate{December}{2015}

\bibliography{biblio.bib}
%% ------------------------------------------------------------------- [ Begin ]
\begin{document}
\maketitle
%% ------------------------------------------------------------- [ Frontmatter ]
\frontmatter
%%
\begin{theabstract}
An abstract of words.
\end{theabstract}

\makedeclaration{} % if custom declaration use declaration environment.
\makecopyrightnoembargo{} % Other copyright statements to be supported in future.


\begin{acknowledgements}
I acknowledge people.

\theAcknowledgementStamp{St Andrews}
\end{acknowledgements}

\cleardoublepage{}

\begin{dedication}
Something snazzy like:
\vspace{3em}

voor het schuim op het bier,\\
voor de nachten plezier\\
voor alle liedjes die ooit zijn bedacht\\
voor het haar op mijn hoofd,\\
voor het vuur dat nooit dooft\\
voor die foto waarop jij naar mij lacht\\
\textit{Halleluja}~\textsc{Jeroen Kant}
\vfill
\end{dedication}

%% --------------------------------------------------------------------- [ TOC ]
\tableofcontents

%% -------------------------------------------------------------- [ Mainmatter ]
\mainmatter{}

% If you are to use latex please use includes.
\chapter{Introduction}
\nocite{*}
\citeauthor{Shannon1946} wrote a cool paper \citep{Shannon1946}. The security of public key protocols has been studied before \citep{Dolev1983}. \citeauthor{Shamir1979} introduces in \citet{Shamir1979} how to share secrets.
There are many interesting papers outthere \citet{Needham1978,Rivest1978,Merkle1978,Diffie1976,Shannon1949,Ziv1977, Huffman1952, Hamming1950}

\begin{fncycomment}
This should only be seen in draft mode.
\end{fncycomment}

\begin{quote}
We few, we happy few, we band of brothers\anote[Bob]{Look Ma' a todo note};
For he to-day that sheds his blood with me
Shall be my brother; be he ne'er so vile,
This day shall gentle his condition:
And gentlemen in England now a-bed
Shall think themselves accursed they were not here,
And hold their manhoods cheap whiles any speaks
That fought with us upon Saint Crispin's day.
\end{quote}

\lipsum{}

\chapter{Method}

Please consider use of the \texttt{csquotes} package, it is loaded by default and helps you with \enquote{quotes}.

\lipsum{}

\begin{theorem}\label{th:a}
Let $a$ and $b$ be in $\mathbb{N}$. Then there exist integers $u$ and $v$, such that
\[
gcd(a,b) = ua+vb.
\]
In particular, if $a$ and $b$ are coprime, there exist integers $u$ and $v$, such tat
\[
ua+vb=1.
\]
\end{theorem}

\begin{lemma}\label{lem:a}
Let $d$ divide a product $ab$ and let the gcd of $d$ and $a$ be 1. Then $d$ divides $b$.
\end{lemma}

\begin{proof}
Since $\mathrm{gcd}(d,a)=1$, Theorem~\ref{th:a} implies that $xd + ya = 1$, for some integers $x$ and $y$.
So, $xdb + yab = b$. Since $d$ divides $ab$, it follows that $d$ also divides $xdb + yab$ which equals $b$.
\end{proof}

\begin{corollary}
let $p$ be prime and let $p$ divide $\prod_{i=1}^{k}a_{i}$, where $a_{i}\in\mathbb{Z}, 1\geq i\geq k$. Then $p$ divides at least one of the factors $a_{i}, 1\geq i\geq k$.
\end{corollary}

\begin{proof}
Use Lemma~\ref{lem:a} and induction on $k$.
\end{proof}

\noindent
A random labelled equation.
\begin{equation}\label{eq:a}
a_{i}\equiv a_{j} \pmod{m}\Rightarrow i=j
\end{equation}

\begin{theorem}[Fermat's Little Theorem]
Let $p$ be a prime number and let $a$ be any integer. Then
\[
a^{p}\equiv a\pmod{p}
\]
\end{theorem}

\begin{definition}
A function $f:\mathbb{N}\rightarrow\mathbb{N}$ is said to be \emph{multiplicative}, if for every pair of positive integers $m$ and $n$
\[
\mathrm{gcd}(m,n)=1\Rightarrow f(m,n)=f(m)f(n)
\]
\end{definition}

\begin{lemma}
Euler's Totient function $\phi(m)$ is multiplicative.
\end{lemma}

\begin{example}
An example
\end{example}

\begin{exercise}
An exercise
\end{exercise}

\begin{problem}
A problem
\end{problem}

\begin{solution}
A solution
\end{solution}

\begin{proposition}
A proposition
\end{proposition}


\begin{definition}
A definition
\end{definition}

\begin{note}
\lipsum[1]
\end{note}

\begin{remark}
\lipsum[1]
\end{remark}

\chapter{Results}

\lipsum{}
\chapter{Discussion}

\begin{quote}
We few, we happy few, we band of brothers\anote[Bob]{Look Ma' a todo note};
For he to-day that sheds his blood with me
Shall be my brother; be he ne'er so vile,
This day shall gentle his condition:
And gentlemen in England now a-bed
Shall think themselves accursed they were not here,
And hold their manhoods cheap whiles any speaks
That fought with us upon Saint Crispin's day.
\end{quote}

\lipsum[2-4]

\chapter{Conclusion}

\lipsum{}

%% ------------------------------------------------------------- [ Back Matter ]
\backmatter
%% ------------------------------------------------------------ [ Bibliography ]
\printbibliography{} % <- biblatex has cool features, I suggest you use them.
%% ----------------------------------------------------------- [ List o' Stuff ]
\cleardoublepage{}
\listoftheorems[ignore={note,remark}]

\end{document}
%% --------------------------------------------------------------------- [ EOF ]
